\section{Taustaa}
\label{sec:taustaa}
\hyperlink{index}{Takaisin sisällysluetteloon}

\subsection{HTML5}
HTML5 on uusin versio HyperText Markup Language-kielestä. Yksi HTML5:n tuomia uudistuksia on erinäisten media-formaattien sisäistäminen itse spesifikaatioon erinäisten rajapintojen kautta\hyperlink{}{\cite{html5diff}}. Näin ollen sisällöntuottajien ja käyttäjien ei tarvitse huolehtia kolmansien osapuolten liitännäisistä kuten Flash. Tällöin sisällöntuottaja voi olla varma siitä että kaikke ne joilla on spesifikaatiota noudattava selain pääsevät käsiksi hänen tuottamaan sisältöön. Täten alustojen erilaisuuksien ei pitäisi ainakaan teoriassa vaikuttaa sovellukseen, jolloin sisällöntuottaja voi keskittyä luomaan sisältöä ainoastaan yhdelle alustalle.

\subsection{WebGL}
WebGL on JavaScript-rajapinta HTML5:n canvas-elementtiä varten. WebGL perustuu pitkälti OpenGL ES:ään (OpenGL Embedded Systems)\hyperlink{}{\cite{webgl_specification}}. Rajapinta sallii näytönohjaimen käytön JavaScript-koodista ilman että sovellusohjelmoijan tarvitsee tietää tarkalleen mistä laittesta on kyse: riittää että kyseinen laite toimii WebGL:n kanssa\hyperlink{}{\cite{webgl_supported}}. Näytönohjaimen käyttö mahdollistaa siis raskaan laskennan siirtämisen pois suorittimesta: fysiikan mallinnus ja monimutkaisen grafiikan piirtäminen ovat tällöin mahdollisia.

\subsection{Pelinkehittäjän tarpeet}
Erityisesti nopeita reaktioita vaativien pelien pelattavuus kärsii huomattavasti liian pienillä ruudun päivitysnopeuksilla: ero pelaajan pelaamiskyvyssä korkeiden ja matalien piirtotaajuuksien välillä on huomattava\hyperlink{}{\cite{claypool_fps}}. Näin ollen pelin sunnittelijan kannattaa rajata pelinsä sisältö sellaiseksi että se pystytään esittämään tarvittavan sujuvassa muodossa. Tällöin joidenkin pelityyppien näyttävyys saattaa rajautua huomattavasti jos ne ovat ollenkaan toteutettavissa. Jos WebGL on toimiva työkalu, se lisäisi pelinkehittäjien ilmaisuvoimaa. Paljon laskentatehoa vaativat pelityypit ovat tähän mennessä olleet pakostakin natiivisovelluksia: jos WebGL toimii hyvin, se mahdollistaisi helpomman ja halvemman kehitystyön tällaisille peleille.