\begin{fiabstract}
\raggedright
Tässä kandidaatintyössä tarkastellaan HTML5-specifikaation WebGL-rajapintaa pelinkehittäjän näkökulmasta. Työssä tutkitaan kirjallisuuskatsauksen avulla WebGL-rajapinnan erityispiirteitä ja soveltuvuuksia.

Aikaisemmista tutkimuksista havaitaan että pelien graafisen esityksen laadulla on vaikutus intensiivisten pelien pelattavuuteen. Varsinkin ruudun päivitystiheyden huomataan haittaavan pelattavuutta ollessaan matala. 

WebGL on huomattavasti nopeampi kuin Canvas 2D Context-toteutukset ja selvästi hitaampi kuin natiivi OpenGL/C++-toteutus. Mittaustuloksia tutkimalla huomataan eron johtuvan pääosin JavaScript-koodin hitaudesta.

Asynkronista lataamista ja matriisilaskentaa ei ole toteutettu itse kirjastoon. Todetaan että puutteet on joko korjattava itse tai on käytettävä ulkoista kirjastoa. Itse koodi on monimutkaisempaa kuin Canvas 2D Context-koodi, mutta ei eroa suuresti natiivista OpenGL-koodista.

WebGL-sovellusten huomataan myös olevan riippuvaisempia alustoista kuin Canvas 2D Context-sovellukset. Eri selainten välillä on toiminnallisuuseroja. Erilaiset selain-näytönohjain yhdistelmät voivat myös olla kokonaan toimimattomia. Todetaan että WebGL on alustasta riippumattomampi kuin OpenGL-sovellus. Kuitenkin WebGL-kehittäjän kannattaa huomioida että sovellus ei yllä yksinkertaisemman Canvas 2D Context-sovelluksen alustariippumattomuuteen.

WebGL-sovellukset toimivat parhaiten korkeatasoisen natiivin ja yksinkertaisen verkkosovelluksen välimaastossa. Kehittäjän ei suositella pyrkiä natiivin sovelluksen suorituskykyyn eikä myöskään olettaa yksinkertaisemman pelin alustariippumattomuutta.  
%
%Tiivistelmätekstiä tähän (\languagename). Huomaa, että tiivistelmä tehdään %vasta kun koko työ on muuten kirjoitettu.
\end{fiabstract}