% ---------------------------------------------------------------------
% -------------- PREAMBLE ---------------------------------------------
% ---------------------------------------------------------------------
\documentclass[12pt,a4paper,finnish,oneside]{article}
%\documentclass[12pt,a4paper,finnish,twoside]{article}
%\documentclass[12pt,a4paper,finnish,oneside,draft]{article} % luonnos, nopeampi

% Valitse 'input encoding':
%\usepackage[latin1]{inputenc} % merkistökoodaus, jos ISO-LATIN-1:tä.
\usepackage[utf8]{inputenc}   % merkistökoodaus, jos käytetään UTF8:a
% Valitse 'output/font encoding':
%\usepackage[T1]{fontenc}      % korjaa ääkkösten tavutusta, bittikarttana
\usepackage{ae,aecompl}       % ed. lis. vektorigrafiikkana bittikartan sijasta
% Kieli- ja tavutuspaketit:
\usepackage[english,swedish,finnish]{babel}
% Kurssin omat asetukset aaltosci_t.sty:
\usepackage{aaltosci_t}
% Jos kirjoitat muulla kuin suomen kielellä valitse:
%\usepackage[finnish]{aaltosci_t}           
%\usepackage[swedish]{aaltosci_t}           
%\usepackage[english]{aaltosci_t}           
% Muita paketteja:
\usepackage{alltt}
\usepackage{amsmath}   % matematiikkaa
\usepackage{calc}      % käytetään laskurien (counter) yhteydessä (tiedot.tex)
\usepackage{eurosym}   % eurosymboli: \euro{}
\usepackage{url}       % \url{...}
\usepackage{listings}  % koodilistausten lisääminen
\usepackage{algorithm} % algoritmien lisääminen kelluvina
\usepackage{algorithmic} % algoritmilistaus
\usepackage{hyphenat}  % tavutuksen viilaamiseen liittyvä (hyphenpenalty,...)
\usepackage{supertabular,array}  % useampisivuinen taulukko
\usepackage{hyperref}   % linkkejä
\usepackage{graphicx}   % kuvia
\usepackage[font=footnotesize,labelfont=bf]{caption}   % Kuvatekstejä

% Koko dokumentin kattavia asetuksia:

% Tavutettavia sanoja:
%\hyphenation{vää-rin me-ne-vi-en eri-kois-ten sa-no-jen tavu-raja-ehdo-tuk-set}
% Huomaa, että ylläoleva etsii tarkalleen kyseisiä merkkijonoja, eikä
% ymmärrä taivutuksia. Paikallisesti tekstin seassa voi myös ta\-vut\-taa.

% Rangaistaan tavutusta (ei toimi?! Onko hyphenat-paketti asennettu?)
\hyphenpenalty=10000   % rangaistaan tavutuksesta, 10000=ääretön
\tolerance=1000        % siedetään välejä riveillä
% titlesec-paketti auttaa, jos tämän mukana menee sekaisin

% Tekstiviitteiden ulkoasu.
% Pakettiin natbib.sty/aaltosci.bst liittyen katso esim. 
% http://merkel.zoneo.net/Latex/natbib.php
% jossa selitykset citep, citet, bibpunct, jne.
% Valitse alla olevista tai muokkaa:
%\bibpunct{(}{)}{;}{a}{,}{,}    % a = tekijä-vuosi (author-year)
\bibpunct{[}{]}{;}{n}{,}{,}    % n = numero [1],[2] (numerical style)

% Rivivälin muuttaminen:
\linespread{1.24}\selectfont               % riviväli 1.5
%\linespread{1.24}\selectfont               % riviväli 1, kun kommentoit pois

% ---------------------------------------------------------------------
% -------------- DOCUMENT ---------------------------------------------
% ---------------------------------------------------------------------

\begin{document}

% -------------- Tähän dokumenttiin liittyviä valintoja  --------------

%\raggedright         % Tasattu vain vasemmalta, ei tavutusta
\input{makroja}       % Haetaan joitakin makroja

% Kieli:
% Kielesi, jolla kandidaatintyön kirjoitat: finnish, swedish, english.
% Tästä tulee mm. tietyt otsikkonimet ja kuva- ja taulukkoteksteihin 
% (Kuva, Figur, Figure), (Taulukko, Tabell, Table) sekä oikea tavutus.
\selectlanguage{finnish}
%\selectlanguage{swedish}
%\selectlanguage{english}

% Sivunumeroinnin kanssa pieniä ristiriitaisuuksia.
% Toimitaan pääosin lähteen "Kirjoitusopas" luvun 5.2.2 mukaisesti.
% Sivut numeroidaan juoksevasti arabialaisin siten että 
% ensimmäiseltä nimiölehdeltä puuttuu numerointi.
\pagestyle{plain}
\pagenumbering{arabic}
% Muita tapoja: kandiohjeet: ei numerointia lainkaan ennen tekstiosaa
%\pagestyle{empty}
% Muita tapoja: kandiohjeet: roomalainen numerointi alussa ennen tekstiosaa
%\pagestyle{plain}
%\pagenumbering{roman}        % i,ii,iii, samalla alustaa laskurin ykköseksi

% ---------------------------------------------------------------------
% -------------- Luettelosivut alkavat --------------------------------
% ---------------------------------------------------------------------

% -------------- Nimiölehti ja sen tiedot -----------------------------
%
% Nimiölehti ja tiivistelmä kirjoitetaan seminaarin mukaan joko
% suomeksi tai ruotsiksi (ellei erityisesti kielenä ole englanti). 
% Tiivistelmän voi suomen/ruotsin lisäksi kirjoittaa halutessaan
% myös englanniksi. Eli tiivistelmiä tulee yksi tai kaksi kpl.
%
% "\MUUTTUJA"-kohdat luetaan aaltosci_t.sty:ä varten.

\author{Atte Isopuro}

% Otsikko nimiölehdelle. Yleensä sama kuin seuraavana oleva \TITLE, 
% mutta jos nimiölehdellä tarvetta "kaksiosaiselle" kaksiriviselle
\title{WebGL HTML5-pelinkehittäjän näkökulmasta}
% 2-osainen otsikko:
%\title{\LaTeX{}-pohja kandidaatintyölle \\[5mm] Pitkiä rivejä kokeilun vuoksi.}

% Otsikko tiivistelmään. Jos lisäksi engl. tiivistelmä, niin viimeisin:
\TITLE{WebGL HTML5-pelinkehittäjän näkökulmasta}
%\TITLE{\LaTeX{} för kandidatseminariet med jättelång rubrik som fortsätter och
% fortsätter ännu}
\ENTITLE{WebGL from an HTML5-game developer's perspective}
% 2-osainen otsikko korvataan täällä esim. pisteellä:
%\TITLE{\LaTeX{}-pohja kandidaatintyölle. Pitkiä rivejä kokeilun vuoksi.}

% Ohjaajan laitos suomi/ruotsi ja tarvittaessa eng (tiivistelmän kieli/kielet)
\DEPT{Tietoliikenneohjelmistot}
%\DEPT{Institutionen för informationsnätverk}
% suomi:
%\DEPT{Tietotekniikan laitos}               % T
%\DEPT{Tietojenkäsittelytieteen laitos}     % TKT
%\DEPT{Mediatekniikan laitos}               % ME
% ruotsi:
%\DEPT{Institutionen för datateknik}        % T
%\DEPT{Institutionen för datavetenskap}     % TKT
%\DEPT{Institutionen för mediateknik}       % ME
% englanti:
%\ENDEPT{Department of Computer Science Engineering}     % T
%\ENDEPT{Department of Information and Computer Science} % TKT
%\ENDEPT{Department of Media Technology}                 % ME

% Vuosi ja päivämäärä, jolloin työ on jätetty tarkistettavaksi.
\YEAR{2014}
\DATE{14. helmikuuta 2014}
%\DATE{31. helmikuuta 2011}
%\DATE{Den 31 februari 2011}
\ENDATE{4. huhtikuuta 2014}

% Kurssin vastuuopettaja ja työsi ohjaaja(t)
\SUPERVISOR{Ma professori Tomi Janhunen}
\INSTRUCTOR{Professori Jukka Nurminen}
%\INSTRUCTOR{Ohjaajantitteli Sinun Ohjaajasi, ToinenTitt Matti Meikäläinen}
% DI       // på svenska DI diplomingenjör
% TkL      // TkL teknologie licentiat
% TkT      // TkD teknologie doctor
% Dosentti Dos. // Doc. Docent
% Professori Prof. // Prof. Professor
% 
% Jos tiivistelmä englanniksi, niin:
\ENSUPERVISOR{Professor (pro tem) Tomi Janhunen}
\ENINSTRUCTOR{Professor Jukka Nurminen}
% M.Sc. (Tech)  // M.Sc. (Eng)
% Lic.Sc. (Tech)
% D.Sc. (Tech)   // FT filosofian tohtori, PhD Doctor of Philosophy
% Docent
% Professor

% Kirjoita tänne HOPS:ssa vahvistettu pääaineesi.
% Pääainekoodit TIK-opinto-oppaasta.

\PAAAINE{Ohjelmistotekniikka}
\CODE{T3001}

%\PAAAINE{Ohjelmistotuotanto ja -liiketoiminta}
%\CODE{T3003}
%
%\PAAAINE{Tietoliikenneohjelmistot}
%\CODE{T3005}
%
%\PAAAINE{WWW-teknologiat} % vuodesta 2010
%\CODE{IL3012}
%
%\PAAAINE{Mediatekniikka} % vuoteen 2010, kts. seur.
%\CODE{T3004}
%
%\PAAAINE{Mediatekniikka} % vuodesta 2010, kts. edell.
%\CODE{IL3011}
%
%\PAAAINE{Tietojenkäsittelytiede} % vuodesta 2010
%\CODE{IL3010}
%
%\PAAAINE{Informaatiotekniikka} % vuoteen 2010
%\CODE{T3006}
%
%\PAAAINE{Tietojenkäsittelyteoria} % vuoteen 2010
%\CODE{T3002}
%
%\PAAAINE{Ohjelmistotekniikka}
%\CODE{T3001}

% Avainsanat tiivistelmään. Tarvittaessa myös englanniksi:

\KEYWORDS{webgl, html5, peli, pelinkehitys, suorituskyky, ongelmia, edut}
\ENKEYWORDS{webgl, html5, game, development, performance, issues, problems, advantages}

% Tiivistelmään tulee opinnäytteen sivumäärä.
% Kirjoita lopulliset sivumäärät käsin tai kokeile koodia. 
%
% Ohje 29.8.2011 kirjaston henkilökunnalta:
%   - yhteissivumäärä nimiölehdeltä ihan loppuun
%   - "kaikkien yksinkertaisin ja yksiselitteisin tapa"
%
% VANHA // Ohje 14.11.2006, luku 4.2.5:
% VANHA // - sivumäärä = tekstiosan (alkaen johdantoluvusta) ja 
% VANHA //  lähdeluettelon sivumäärä, esim. "20"
% VANHA // - jos liitteet, niin edellisen lisäksi liitteiden sivumäärä,
% VANHA //  tyyli "20 + 5", jossa 5 sivua liitteitä 
% VANHA // - HUOM! Tässä oletuksena sivunumerointi alkaa nimiölehdestä 
% VANHA //  sivunumerolla 1. %   Toisin sanoen, viimeisen lähdeluettelosivun 
% VANHA //  sivunumero EI ole sivujen määrä vaan se pitää laskea tähän käsin

\PAGES{TODO: Kirjoita tähän lopuksi oikea määrä, tässä esimerkissä 23}
%\PAGES{23}  % kaikki sivut laskettuna nimiölehdestä lähdeluettelon tai 
             % mahdollisten liitteiden loppuun. Tässä 23 sivua

%\thispagestyle{empty}  % nimiölehdellä ei ole sivunumerointia; tyylin mukaan ei tehdäkään?!

\maketitle             % tehdään nimiölehti

% -------------- Tiivistelmä / abstract -------------------------------
% Lisää abstrakti kandikielellä (ja halutessasi lisäksi englanniksi).
\begin{fiabstract}
\raggedright
Tässä kandidaatintyössä tarkastellaan HTML5-specifikaation WebGL-rajapintaa pelinkehittäjän näkökulmasta. Työssä tutkitaan kirjallisuuskatsauksen avulla WebGL-rajapinnan erityispiirteitä ja soveltuvuuksia.

Aikaisemmista tutkimuksista havaitaan että pelien graafisen esityksen laadulla on vaikutus intensiivisten pelien pelattavuuteen. Varsinkin ruudun päivitystiheyden huomataan haittaavan pelattavuutta ollessaan matala. 

WebGL on huomattavasti nopeampi kuin Canvas 2D Context-toteutukset ja selvästi hitaampi kuin natiivi OpenGL/C++-toteutus. Mittaustuloksia tutkimalla huomataan eron johtuvan pääosin JavaScript-koodin hitaudesta.

Asynkronista lataamista ja matriisilaskentaa ei ole toteutettu itse kirjastoon. Todetaan että puutteet on joko korjattava itse tai on käytettävä ulkoista kirjastoa. Itse koodi on monimutkaisempaa kuin Canvas 2D Context-koodi, mutta ei eroa suuresti natiivista OpenGL-koodista.

WebGL-sovellusten huomataan myös olevan riippuvaisempia alustoista kuin Canvas 2D Context-sovellukset. Eri selainten välillä on toiminnallisuuseroja. Erilaiset selain-näytönohjain yhdistelmät voivat myös olla kokonaan toimimattomia. Todetaan että WebGL on alustasta riippumattomampi kuin OpenGL-sovellus. Kuitenkin WebGL-kehittäjän kannattaa huomioida että sovellus ei yllä yksinkertaisemman Canvas 2D Context-sovelluksen alustariippumattomuuteen.

WebGL-sovellukset toimivat parhaiten korkeatasoisen natiivin ja yksinkertaisen verkkosovelluksen välimaastossa. Kehittäjän ei suositella pyrkiä natiivin sovelluksen suorituskykyyn eikä myöskään olettaa yksinkertaisemman pelin alustariippumattomuutta.  
%
%Tiivistelmätekstiä tähän (\languagename). Huomaa, että tiivistelmä tehdään %vasta kun koko työ on muuten kirjoitettu.
\end{fiabstract}
\begin{svabstract}
 Ett abstrakt hit 
%(\languagename)
\end{svabstract}

% Edelleen sivunumerointiin. Eräs ohje käskee aloittaa sivunumeroiden
% laskemisen nimiösivulta kuitenkin niin, että sille ei numeroa merkitä
% (Kauranen, luku 5.2.2). Näin ollen ensimmäisen tiivistelmän sivunumero
% on 2. \maketitle komento jotenkin kadottaa sivunumeronsa.
\setcounter{page}{2}    % sivunumeroksi tulee 2
\newpage                       % pakota sivunvaihto

% -------------- Sisällysluettelo / TOC -------------------------------
\label{index}
\hypertarget{index}{}   % hyperlink vaatii toimiakseen, pelkkä label ei riitä
\tableofcontents

\label{pages:prelude}
\clearpage

\section{Johdanto}
\label{sec:johdanto}

Tämä kandidaatintyö käsittelee HTML5:n WebGL-ohjelmointirajapintaa pelinkehittäjän näkökulmasta. 

HTML5-spesifikaatio on mahdollistanut kaikenlaisen median esittämisen verkossa ilman että käyttäjän tarvitsee asentaa erillisiä lisäosia. Yksi näistä uusista rajapinnoista on WebGL jonka kautta verkkosivun on mahdollista käyttää tietokoneen näytönohjainta piirtämiseen.

Erityisesti monimutkaista ja tarkkaa interaktiota vaativat pelit tarvitsevat nopeaa grafiikan piirtämistä ruudulle. Alustasta riippumaton grafiikkaohjelmointi on erittäin kiinnostavaa tästä näkökulmasta. 

Kirjallisuudessa on käsitelty WebGL:n käyttömahdollisuuksia erinäisillä aloilla. Useimmat tällaiset käsittelyt ovat kuitenkin yksittäisiä sovelluksia: yleistä arviota WebGL:n eduista ja haitoista pelinkehityksessä ei ole tehty. 

Tämän kandidaatintyön tavoitteena on kerätä pelinkehittäjän kannalta hyödyllisiä havaintoja WebGL:stä. Työssä pyritään tunnistamaan sellaiset WebGL:n ominaispiirteet jotka  pelinkehittäjän tulisi ottaa huomioon.
Tämä työ ei tule käsittelemään muita selaimen grafiikka-esitykseen liittyviä teknologioita, kuten Canvas 2D context tai Flash. Teknologioita verrataan WebGL:ään, mutta niiden erityispiirteitä ei tutkita tässä työssä tarkemmin. Työssä ei myöskään pyritä toteamaan WebGL:n paremmuudesta mitään konkreettista. Löytöjen perusteella peinkehittäjän on yhä itse päätettävä mikä teknologia soveltuu hänen projektiinsa parhaiten. 

Työ toteutetaan kirjallisuuskatsauksena. Aineisto koostuu tieteellisistä artikkeleista ja kirjoista jotka käsittelevät WebGL:ää. Erityisesti viitataan teksteihin joissa on konkreettisten toteutusten yhteydessä havaittu WebGL:n ominaispiirteitä.

Ensiksi alustetaan aiheen taustaa: esitellään HTML5 ja grafiikan piirtäminen. Seuraavaksi selvitetään tarkemmin pelinkehittäjän tarpeet sekä WebGL-spesifikaation tarjoamat edut. Tämän jälkeen eritellään tutkimusmenetelmät: lähteiden rajauskriteerit selvitetään ja kerrotaan kuinka ne on määrätty luotettaviksi.
Lopuksi aineistosta kerätty tieto kootaan eheäksi kokonaisuudeksi, josta viimein tullaan lopullisiin johtopäätöksiin.
\clearpage                     % luku loppuu, loput kelluvat tänne, sivunv.

\section{Aiheeseen liittyvät teknologiat}
\label{sec:taustaa}

\subsection{Selainteknologiat}
Seuraavaksi esitellään tämän työn kannalta keskeisimmät verkkoselaimissa käytetyt teknologiat. 

\subsubsection{HTML5}
HTML5 on uusin versio \textbf{H}yper\textbf{T}ext \textbf{M}arkup \textbf{L}anguage-kielestä\cite{htmlapis}. Yksi HTML5-kielen tuomia uudistuksia on erinäisten media-formaattien sisäistäminen itse spesifikaatioon erinäisten rajapintojen kautta\cite{html5diff}. Näin ollen sisällöntuottajien ja käyttäjien ei tarvitse huolehtia kolmansien osapuolten liitännäisistä kuten Flash. Tällöin sisällöntuottaja voi olla varma siitä että kaikki ne joilla on spesifikaatiota noudattava selain pääsevät käsiksi hänen tuottamaan sisältöön. Täten alustojen erilaisuuksien ei pitäisi ainakaan teoriassa vaikuttaa sovellukseen, jolloin sisällöntuottaja voi keskittyä luomaan sisältöä ainoastaan yhdelle alustalle.

\subsubsection{JavaScript}
JavaScript-kieli on laajasti käytössä oleva ohjelmointikieli. Kieli on toteutus ECMAScript-standardista\cite{ecmascript} jota käytetään pääasiassa verkkosivujen ohjelmointiin\cite{mdnjs}. Kieli mahdollistaa verkkosivujen reaaliaikaisen muuttamisen, animoinnit sekä käyttäjän interaktiot. Useimmat HTML5-kielen määrittämät rajapinnat ovat juuri JavaScript-ohjelmia varten tarkoitettuja.

\subsubsection{Canvas-elementti}
\texttt{<canvas>}-elementti on HTML5-spesifikaation määrittelemä elementti jonka voi sisällyttää verkkosivulleen HTML5-koodissa. Elementin tarkoitus on toimia piirtoalustana: ohjelmoija voi rajapintojen kautta piirtää \texttt{<canvas>}-elementille grafiikkaa. \texttt{<canvas>}-elementin piirtämiseen tarkoitetut rajapinnat ovat tämän työn käsittelemä WebGL-rajapinta ja yksinkertaisempi Canvas 2D Context.\cite{canvas_spec}

\subsection{Intensiivinen grafiikka ja näytönohjaimen rajapinnat}
Intensiivisellä grafiikalla tarkoitetaan tässä yhteydessä grafiikkaa jonka piirtäminen vaatii huomattavan määrän laskentatehoa. Esimerkiksi yksittäisen digitaalisen valokuvan piirtäminen ruudulle ei ole intensiivistä grafiikkaa. Fysikaalisen ilmiön (kuten veden) simuloiminen ja piirtäminen ruudulle taas on hyvin intensiivistä ja voi veden liikkeistä ja koostumuksesta riippuen vaatia suuria määriä laskentatehoa. 3D-grafiikka on melkein aina paljon raskaampaa kuin 2D-grafiikka. Kuitenkin piirron intensiivisyys riippuu ainoastaan laskennan määrästä. 2D-kuva jossa on paljon objekteja ja korkea resoluutio voi myös olla raskas piirtää. Intensiivinen grafiikka ei siis rajoitu ainoastaan 3D-grafiikkaan.

\subsubsection{Peligrafiikka}
\mbox{Pelialalla intensiivinen grafiikka yhdistyy useimmiten fysiikan simulointiin}, kolmiuloitteisiin ympäristöihin ja monimutkaisiin esineisiin pelimaailmassa. Vaikka tällaisia asioita voidaan periaatteessa simuloida tietokoneen pääsuorittimella, teho jää nopeasti riittämättömäksi. Liian intensiivinen grafiikka hidastaisi konetta sen verran että käyttökokemus kärsii. Graafisesti monimutkaisten pelien laskenta on näin ollen siirrettävä yleispätevästä suorittimesta näytönohjaimeen, joka on graafiseen laskentaan erikoistunut laitekomponentti.

\subsubsection{Näytönohjain}
Näytönohjain on tietokoneeseen liitettävä erillinen lisälaite, jolla on oma muisti ja oma graafiseen laskentaan erikoistunut suoritin. Näin ollen laskenta joka suoritetaan näytönohjaimessa ei käytä yhtä paljon tietokoneen resursseja. Näytönohjaimen käyttäminen koodista on kuitenkin monimutkaisempaa kuin pelkän suorittimen käskyttäminen. Ohjelmoijan on varmistettava, että käskyjen lisäksi myös tarvittava muisti siirretään näytönohjaimelle. Näytönohjain on useimmiten myös erillinen komponentti, joten käyttöjärjestelmä tarvitsee erillisen ajurin tai ohjaimen voidakseen käyttää sitä. Näitä ajureita voidaan käyttää erityisillä grafiikkaohjelmointiin tarkoitetuilla rajapinnoilla.

\subsubsection{OpenGL-rajapinta}
OpenGL (\textbf{Open} \textbf{G}raphics \textbf{L}ibrary) on alusta- ja kieliriippumaton ohjelmointirajapinta tietokoneiden näytönohjaimille\cite{opengl4}. Ohjelmoija voi käyttää OpenGL-kirjastoa tehdäkseen kutsuja näytönohjaimelle, ilman että hänen tarvitsee käyttää näytönohjaimen omaa, matalan tason käskykantaa. Vastaava teknologia on DirectX\cite{directx}, Microsoftin ylläpitämä lisensioitava rajapinta-kirjasto, joka on tarkoitettu yksinomaan Windows-käyttöjärjestelmälle.

\subsubsection{OpenGL ES-rajapinta}
OpenGL ES (\textbf{E}mbedded \textbf{S}ystems) on OpenGL-kirjaston versio joka on tarkoitettu käytettäväksi sulautetuissa järjestelmissä, kuten pelikonsoleissa, puhelimissa ja tabletti-tietokoneissa. OpenGL ES on rajatumpi versio täydestä OpenGL-kirjastosta, jotta se olisi mahdollisimman laajasti käytettävä.\cite{opengles} 

\subsubsection{WebGL-rajapinta}
WebGL (\textbf{Web} \textbf{G}raphics \textbf{L}ibrary) on JavaScript-rajapinta \texttt{<canvas>}-elementtiä varten. WebGL perustuu pitkälti OpenGL ES:ään\cite{webgl_specification}. Näin ollen WebGL on samalla lailla rajattu: näitä rajallisuuksia tarkastellaan tarkemmin kappaleessa \ref{subsec:piirteet}. Rajapinta sallii näytönohjaimen käskyttämisen JavaScript-koodista. JavaScript-ohjelmoija voi siis piirtää \texttt{<canvas>}-elementille kuvia WebGL-rajapinnan kautta samalla tavoin kuin natiivisovellus tekisi esimerkiksi C++-kieltä ja OpenGL-rajapintaa käyttäen. Canvas 2D Context-rajapinta voi ainoastaan käyttää suoritinta, eikä sitä ole tarkoitettu monimutkaisen 3D-grafiikan piirtoon.
\clearpage

\section{Sovelluksia}
\label{sec:sovelluksia}
\hyperlink{index}{Takaisin sisällysluetteloon}
\subsection{Tuoteesittelyä}
\subsection{Lääketieteellistä kuvantamista}
\subsection{Pelejä}
\clearpage

\section{Tutkimusmateriaali}
\label{sec:materiaali}
\hyperlink{index}{Takaisin sisällysluetteloon}
\subsection{Suorituskyky}
Selkein WebGL:n tarjoama etu on suorituskyky: mahdollisuus suorittaa laskentaa näytönohjaimessa suorittimen sijasta on huomattavasti nopeampaa: Canvas 2D Context ja Flash ovat vertailussa selvästi hitaampia\cite{hoetzlein}. WebGL pärjää myös yllättävän hyvin natiivia toteutusta vastaan. C++-pohjainen yksinkertainen OpenGL-sovellus on WebGL:ää hitaampi: natiivi toteutus täytyy optimoida ennen kuin se on tehokkaampi kuin WebGL-toteutus\cite{hoetzlein}. Ero ei suurimmaksi osaksi kuitenkaan johdu WebGL:n ja natiivin OpenGL:n välisistä eroista. Koska WebGL:ää kutsutaan verkkosivun koodista, selaimen on suoritettava JavaScript-koodia jokaista piirtoa varten. JavaScript-koodi on muutamia erikoistapauksia lukuunottamatta paljon hitaampaa kuin C++-koodi\cite{smedberg}. Täten suurin osa WebGL:n piirtämisajasta kuluu JavaScriptin kääntämiseen ja ajamiseen\cite{hoetzlein}.

Pelinkehittäjän kannattaa siis ottaa huomioon että WebGL on huomattavasti hitaampi kuin optimoitu natiivi sovellus, mikä oli odotettavissa. Huomionarvoisempaa on kuitenkin että WebGL ei itse ole pääsyyllinen kyseiseen eroon: WebGL-koodin optimoiminen ei merkittävästi vaikuttanut piirtonopeuteen\cite{hoetzlein}. On huomionarvoista että WebGL-grafiikan optimoimisessa ei välttämättä kannata keskittyä itse WebGL-koodiin: JavaScript-koodin optimoiminen voi tuottaa paljon suuremmat hyödyt. Näin ollen JavaScript-kokemus voi kompensoida puutteita kehittäjän WebGL-kokemuksessa, mikä saattaa huomattavasti helpottaa teknologian käyttöönottoa pelinkehittäjien keskuudessa.

\subsection{Kirjaston kattavuus}
Pelinkehittäjän on aina hyvä tietää mihin valittu teknologia pystyy ja mitä puutteita sillä on. Näin ollen on aiheellista katsastaa WebGL:n funktiokirjastoa, jossa on muutamia puutteita\cite{dibenedettoSpider}:

WebGL on matalan tason kirjasto joka oletusarvoisesti toimii sekä tietokoneilla että mobiililaitteilla. Tästä syystä siitä saattaa puuttua toiminallisuutta joka löytyy oletusarvona suuremmista kirjastoista.
Yksi tärkeä ominaisuus nykyaikaisissa grafiikkakirjastoissa on asynkroninen lataaminen. Kuvaa piirrettäessä tietokoneen on haettava muistista erinäistä tietoa: tekstuureita, malleja ym. Jos tällaiset tiedot pitää ladata yksi kerrallaan ohjelma saattaa lakata reagoimasta käyttäjään kunnes kaikki tarvittava tieto on ladattu. Asynkroninen lataaminen sallii eri osien lataamisen samanaikaisesti, jolloin suoritin voi samalla reagoida käyttäjään. JavaScript salli tällaisen asynkronisuuden ainoastaan Web Workers rajapinnan kautta\cite{htmlwebworkers}, joten ominaisuutta ei löydy oletuksena WebGL:ää käyttäessä.

Yksi HTML5:n hyödyllisistä ominaisuuksista ovat elementti-kohtaiset rajapinnat. Esimerkiksi \begin{verbatim}<img>\end{verbatim}elemetin JavaScript-rajapinnasta löytyy \begin{verbatim}onload\end{verbatim}-funktio, joka sallii kehittäjän määritellä mitä tehdään kun kyseisen elementin määrittelemä kuva on ladattu valmiiksi. JavaScriptissä tai WebGL:ssä ei ole tällaista toiminnallisuutta 3D-objekteille: HTML5 ei määrittele tällaisille objekteille erityisiä elementtejä.

Tärkeä asia pitää mielessä WebGL:ssä on että se perustuu OpenGL ES:ään, joka vuorostaan sisältää vain osan täysimittaisen OpenGL-kirjaston toiminnallisuudesta (kuten OpenGL 3.0)\cite{khronosopenwebgldiff}. Näin ollen kehittäjä joka on kokenut OpenGL-koodaaja ei välttämättä ole niin tehokas kuin voisi toivoa: WebGL:n rajallisuus OpenGL:ään verrattuna saattaa vaatia sopeutumista.

Huomattavin puute WebGL:ssä muihin grafiikkakirjastoihin verrattuna on lineaarialgebran puuttuminen. 3D-grafiikan piirtämisessä käytetään huomattavia määriä erilaista matriisilaskentaa. Näitä funktioita ei löydy WebGL:stä, joten kehittäjän on joko toteutettava ne itse tai käytettävä kolmannen osapuolen kehittämää kirjastoa.
\subsection{Alustariippumattomuus}
\clearpage

\section{Arviointi}
\label{sec:arviointi}
\hyperlink{index}{Takaisin sisällysluetteloon}
\clearpage

\section{Johtopäätökset}
\label{sec:johtopaatokset}
\hyperlink{index}{Takaisin sisällysluetteloon}
\clearpage

%\input{luku2}                  % tässä tyylissä ei sivunvaihtoja lukujen
%\input{luku3}                  %   välillä. Toiset ohjaajat haluavat 
%\input{luku4}                  %   sivunvaihdot.

\label{pages:text}
\clearpage                     % luku loppuu, loput kelluvat tänne, sivunvaihto
%\newpage                       % ellei ylempi tehoa, pakota lähdeluettelo 
                               % alkamaan uudelta sivulta

% -------------- Lähdeluettelo / reference list -----------------------
%
% Lähdeluettelo alkaa aina omalta sivultaan; pakota lähteet alkamaan
% joko \clearpage tai \newpage
%
%
% Muista, että saat kirjallisuusluettelon vasta
%  kun olet kääntänyt ja kaulinnut "latex, bibtex, latex, latex"
%  (ellet käytä Makefilea ja "make")

% Viitetyylitiedosto aaltosci_t.bst; muokattu HY:n tktl-tyylistä.
\bibliographystyle{aaltosci_t}
% Katso myös tämän tiedoston yläosan "preamble" ja siellä \bibpunct.

% Muutetaan otsikko "Kirjallisuutta" -> "Lähteet"
\renewcommand{\refname}{Lähteet}  % article-tyyppisen
%\renewcommand{\bibname}{Lähteet}  % jos olisi book, report-tyyppinen

% Lisätään sisällysluetteloon
\addcontentsline{toc}{section}{\refname}  % article
%\addcontentsline{toc}{chapter}{\bibname}  % book, report

% Määritä kaikki bib-tiedostot
\bibliography{kandi_lahteet}
%\bibliography{thesis_sources,ietf_sources}

\label{pages:refs}
\clearpage         % erotetaan mahd. liitteet alkamaan uudelta sivulta

% -------------- Liitteet / Appendices --------------------------------
%
% Liitteitä ei yleensä tarvita. Kommentoi tällöin seuraavat
% rivit.

% Tiivistelmässä joskus matemaattisen kaavan tarkempi johtaminen, 
% haastattelurunko, kyselypohja, ylimääräisiä kuvia, lyhyitä 
% ohjelmakoodeja tai datatiedostoja.

% \appendix
% \input{luku_liitteet}

% \label{pages:appendices}

% ---------------------------------------------------------------------

\end{document}
