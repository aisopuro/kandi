\documentclass[12pt,a4paper,finnish,oneside]{article}

% Valitse 'input encoding':
%\usepackage[latin1]{inputenc} % merkistökoodaus, jos ISO-LATIN-1:tä.
\usepackage[utf8]{inputenc}   % merkistökoodaus, jos käytetään UTF8:a
% Valitse 'output/font encoding':
%\usepackage[T1]{fontenc}      % korjaa ääkkösten tavutusta, bittikarttana
\usepackage{ae,aecompl}       % ed. lis. vektorigrafiikkana bittikartan sijasta
% Kieli- ja tavutuspaketit:
\usepackage[finnish]{babel}
% Muita paketteja:
% \usepackage{amsmath}   % matematiikkaa
\usepackage{url}       % \url{...}
\usepackage{hyperref}

% Kappaleiden erottaminen ja sisennys
\parskip 1ex
\parindent 0pt
\evensidemargin 0mm
\oddsidemargin 0mm
\textwidth 159.2mm
\topmargin 0mm
\headheight 0mm
\headsep 0mm
\textheight 246.2mm

\pagestyle{plain}

% ---------------------------------------------------------------------

\begin{document}

% Otsikkotiedot: muokkaa tähän omat tietosi

\title{TIK.kand tutkimussuunnitelma:\\[5mm]
WebGL Pelinkehittäjän Näkökulmasta}

\author{Atte Isopuro\\
Aalto-yliopisto\\
\url{atte.isopuro@aalto.fi}}

\date{\today}

\maketitle

% ---------------------------------------------------------------------

% MUOKKAA TÄHÄN. Jos tarvitset tähän viitteitä, käytä
% tässä dokumentissa numeroviitejärjestelmää komennolla \cite{kahva}.
%
% Paljon kandidaatintöitä ohjanneen Vesa Hirvisalon tarjoama 
% sabluuna. Kursivoidut osat \emph{...} ovat kurssin henkilökunnan
% lisäämiä. 

\textbf{Kandidaatintyön nimi:} WebGL Pelinkehittäjän Näkökulmasta

\textbf{Työn tekijä:} Atte Kaspar Isopuro

\textbf{Ohjaaja:} Jukka Nurminen


\section{Tiivistelmä tutkimuksesta}

Tutkimuksen tarkoituksena on tarkastella WebGL-rajapintaa HTML5-pelinkehittäjän näkökulmasta.

\section{Tavoitteet ja näkökulmat}

Aihetta tutkitaan pelikehittäjän näkökulmasta. Onko WebGL riitävän tehokas verrattuna natiivisti toteutettuun grafiikan piirtämiseen? Asettaako teknologian käyttö rajoituksia pelien toteutukselle?

\section{Tutkimusmateriaali}

Tutkimus perustuu kirjallisuuskatsastukseen. Katsastuksessa keskitytään tieteellisiin artikkeleihin jotka tutkivat WebGL:n suorituskykyä sekä sen käyttöä interaktiivisen grafiikan piirtämisessä. Tulosten perusteella pyritään pääsemään johtopäätökseen WebGL:n erityispiirteistä HTML5-pelinkehityksessä.

WebGL-teknologiasta on tehty paljon eri alojen tutkimuksia: sisällön riittävyyttä pohditaan enemmän kohdassa~\hyperlink{sec:haasteet}{\ref{sec:haasteet}: Haasteet}.

\section{Tutkimusmenetelmät}

Aluksi rajataan aineisto WebGL:ää käsitteleviin tieteellisiin julkaisuihin. Alustava kriteeri on että tekstillä on jotain konkreettista sanottavaa WebGL:stä tai sen avulla toteutetuista tekniikoista.

Tämän jälkeen viitteiden kelpoisuutta tutkitaan tarkistamalla julkaisutietokannoista josko ne viittaavat luotettavaan materiaaliin sekä kuinka usein niihin itseensä on viitattu. Tärkeimpinä työkaluina Scopus ja Google Scholar.

Lopuksi aineisto luetaan huolellisemmin läpi ja kirjataan muistiin sellaiset viitteet jotka ovat tämän tutkimuksen kannalta relevantteja. Tieto, johon viitataan, kirjataan muistiin. Näistä muistiinpanoista koostettava tieto jäsennetään lopulta tarkoituksenmukaiseksi kokonaisuudeksi.

\section{Haasteet}
\label{sec:haasteet}
\hypertarget{sec:haasteet}{Haasteet}

Työn suurin haaste on mahdollinen sisällön riittämättömyys. Viitteitä on paljon, mutta selvästi aiheeseen sopivia ei ole kovin montaa. WebGL:ää on tutkittu paljon muihin tieteenaloihin liittyen: on mahdollista että eteenpäin mennessä sisältö osoittautuu riittämättömäksi. Silloin työn aihetta saattaa joutua laajentamaan jotta saadaan aikaiseksi hyväksyttävä lopputulos.

\section{Resurssit}

Työtä tekee alekirjoittanut, Aalto-Yliopiston opiskelija Atte Isopuro. Aikaa on käytettävissä vaadittava määrä.

Työtä ohjaa Professori Jukka Nurminen.

\section{Aikataulu}

\begin{tabular}{|p{10mm}|p{30mm}|p{110mm}|}
\hline
Vk	&	Raportointi					&	Tehtävää	\\ \hline
7	&	Kandidaatintyö v. 0.1		&	Tutkimus: 8h, Kirjoitus: 8h \\ \hline
8	&	Ei mitään					&	Kielipalaute: 0.3h, Tutkimus: 8h, Kirjoitus: 8h \\ \hline
9	&	Kandidaatintyö v. 0.2		&	Tutkimus: 4h, Kirjoitus: 12h  \\ \hline
10	&	Ei mitään					&	Tutkimus: 4h, Kirjoitus: 12h  \\ \hline
11	&	V2-opponointi				&	Opponointi: 8h, Tutkimus: 4h, Kirjoitus: 12h  \\ \hline
12	&	Kandidaatintyö v. 1.0		&	Kirjoitus: 16h  \\ \hline
13	&	Ei mitään					&	Kirjoitus: 16h  \\ \hline
14	&	Yhteenveto (Ruotsi)			&	Yhteenveto: 8h, Kirjoitus: 16h  \\ \hline
15	&	Ei mitään					&	Kirjoitus: 16h  \\ \hline
16	&	Valmis, kalvot				&	Kalvot: 2h, Kirjoitus: 16h  \\ \hline

\end{tabular}


\section{Esittäminen}

Työllä on alustavasti seuraavanlainen rakenne:

\begin{enumerate}
	\item{Johdanto}
	\item{Taustaa}
	\begin{enumerate}
		\item{HTML5}
		\item{Intensiivinen grafiikka}
	\end{enumerate}
	\item{WebGL:n soveltuvuus pelinkehitykseen}
	\begin{enumerate}
		\item{WebGL:n tarjoamat edut}
		\item{Pelinkehittäjän tarpeet}
	\end{enumerate}
	\item{Tutkimusmenetelmät
	\begin{enumerate}
		\item{Lähteiden löytäminen}
		\item{Luotettavuuden arviointi}
	\end{enumerate}
	\item{Tutkimustiedon arviointi}
	\item{Johtopäätökset}
\end{enumerate}
%

% ---------------------------------------------------------------------
%
% ÄLÄ MUUTA MITÄÄN TÄÄLTÄ LOPUSTA

% Tässä on käytetty siis numeroviittausjärjestelmää. 
% Toinen hyvin yleinen malli on nimi-vuosi-viittaus.

% \bibliographystyle{plainnat}
\bibliographystyle{finplain}  % suomi
%\bibliographystyle{plain}    % englanti
% Lisää mm. http://amath.colorado.edu/documentation/LaTeX/reference/faq/bibstyles.pdf

% Muutetaan otsikko "Kirjallisuutta" -> "Lähteet"
\renewcommand{\refname}{Lähteet}  % article-tyyppisen

% Määritä bib-tiedoston nimi tähän (eli lahteet.bib ilman bib)
\bibliography{lahteet}

% ---------------------------------------------------------------------

\end{document}
